\documentclass[aspectratio=43]{beamer}

% Input
\usepackage[utf8]{inputenc}
\usepackage[T1]{fontenc}
\usepackage{lmodern,charter}
\usepackage[letterspace=100]{microtype}
\usepackage{textcomp}
\usepackage{upquote}

% Beamer
\usepackage{beamercolorthemedove}
\setbeamertemplate{navigation symbols}{}
\setbeamertemplate{frametitle}{\vspace{1cm}\hspace{-1pt}\insertframetitle\vspace{0.33cm}}
\setbeamerfont{frametitle}{size=\huge,series=\bfseries}
\setbeamerfont{normal text}{size=\normalsize,series=\bfseries}
\setbeamertemplate{itemize item}{--}
\setbeamertemplate{itemize subitem}{--}
\setlength{\leftmargini}{12pt}
\setlength{\leftmarginii}{12pt}
\AtBeginDocument{\usebeamerfont{normal text}}

% Tikz
\definecolor{burlywood}{cmyk}{1,1,1,1}
\usepackage{tikz}
\usetikzlibrary{arrows,calc,fit,positioning,shapes,chains}

% Graphics
\usepackage{graphicx}
\usepackage{epstopdf}
\DeclareGraphicsExtensions{.png,.pdf,.eps}

% Various
\usepackage[outline]{contour}
\contourlength{2.00pt}
\usepackage{hyperref}
\usepackage{minted}

% Title
\title{Building instant features with advanced Plone themes}
\author{Asko Soukka <asko.soukka@iki.fi>}

% Begin
\begin{document}

\section{Title}

% Title
%{
%\usebackgroundtemplate{\includegraphics[trim=0 0 0 0,clip,width=0.5\paperwidth]{images/asko.jpg}}
%\begin{frame}[plain,c]
%  \begin{columns}[onlytextwidth,c]
%    \begin{column}{0.45\textwidth}
%    \end{column}
%    \begin{column}{0.45\textwidth}
%      \centering
%      \href{http://iki.fi/asko.soukka/}{Asko Soukka},
%      \href{mailto:asko.soukka@iki.fi}{asko.soukka@iki.fi}\\
%      \href{https://github.com/datakure/}{github.com/datakurre}
%      \par
%      \vspace{1.0cm}
%      \Huge
%      \bfseries
%      Instant features with advanced Plone themes
%      \par
%      \vspace{1.0cm}
%      \large
%      \mdseries
%      \href{https://www.jyu.fi/}{\includegraphics[width=3.1cm]{images/logo.eps}}
%      \par
%    \end{column}
%  \end{columns}
%\end{frame}
%}

\begin{frame}[plain,t]
  \vspace{2em}
  \centering
  \href{http://iki.fi/asko.soukka/}{Asko Soukka},
  \href{mailto:asko.soukka@iki.fi}{asko.soukka@iki.fi}\\
  \href{https://github.com/datakure/}{github.com/datakurre}
  \par
  \vspace{3em}
  \Huge
  \bfseries
  Instant features with advanced Plone themes
  \par
  \vspace{2em}
  \href{https://www.jyu.fi/}{\includegraphics[width=4cm]{images/logo.eps}}
\end{frame}

\section{Motivation}

\begin{frame}[plain,c]
  \huge
  \bfseries
  \centering
  \vspace{1cm}
  What if\ldots
\end{frame}

% one could

% install Plone on cloud with just a click
% get a simple zip file for ordered features
% upload that zip file
% and use the site

% create wall
% upload a lot of images
% publish them

% open the wall for submissions
% visitors can upload submissions
% reviewer can accept or delete submissions

\begin{frame}[plain,t]
  \frametitle{Case of the day:\ Wall of images}
  \stepcounter{beamerpauses}
  \begin{columns}[onlytextwidth,t]
  \begin{column}{0.45\textwidth}
  \begin{itemize}[<+->]
  \setlength{\itemsep}{1em}
  \item Resource bundles
  \item Folderish content type
  \item Custom view template for Wall of images with Masonry.js layout
  \item Workflow for accepting anonymous submissions
  \item Image content type for anonymous submissions
  \end{itemize}
  \end{column}
  \begin{column}{0.45\textwidth}
  \begin{itemize}[<+->]
  \setlength{\itemsep}{1em}
  \item Add permission for restricting anonymous visitor submission
  \item Content rule for thanking about submission
  \item Review portlet for listing the pending submissions
  \item i18n with l10n message catalog
  \end{itemize}
  \end{column}
  \end{columns}
\end{frame}

\begin{frame}[plain,t]
  \frametitle{Python packages vs.\ theme?}
  \stepcounter{beamerpauses}
  \begin{columns}[onlytextwidth,t]
  \begin{column}{0.45\textwidth}
  Python packages
  \vspace{1em}
  \begin{itemize}[<+->]
  \setlength{\itemsep}{1em}
  \item One package for new JS
  \item Another for new types
  \item Customer specific theming of components in theme package
  \item Configuring everything in policy package
  \item Restarting Plone
  \end{itemize}
  \end{column}
  \begin{column}{0.45\textwidth}
  Advanced theme
  \vspace{1em}
  \begin{itemize}[<+->]
  \setlength{\itemsep}{1em}
  \item Everything in a single zipped Plone theme package
  \item Upload through Plone control panel or using npm package
  \href{https://www.npmjs.com/package/plonetheme-upload}{\texttt{plonetheme-upload}}
  \end{itemize}
  \end{column}
  \end{columns}
\end{frame}

\section{Customizing Plone}

\begin{frame}[plain,t]
  \frametitle{Plone customization features}
  \stepcounter{beamerpauses}
  \begin{columns}[onlytextwidth,t]
  \begin{column}{0.45\textwidth}
  \begin{itemize}[<+->]
  \setlength{\itemsep}{1em}
  \item Configurable registry
  \item Structured content types
  \item Content type behaviors
  \item State based workflows
  \item Roles and permissions
  \item All types of portlets
  \end{itemize}
  \end{column}
  \begin{column}{0.45\textwidth}
  \begin{itemize}[<+->]
  \setlength{\itemsep}{1em}
  \item Content rules on events
  \item Restricted templates
  \item Restricted Python scripts
  \item Static frontend resources
  \item Diazo transform rules
  \item \ldots
  \end{itemize}
  \end{column}
  \end{columns}
\end{frame}

\begin{frame}[plain,t]
  \frametitle{Plone customization features}
  \stepcounter{beamerpauses}
  \begin{columns}[onlytextwidth,t]
  \begin{column}{0.45\textwidth}
  \begin{itemize}
  \setlength{\itemsep}{1em}
  \mdseries
  \item Configurable registry
  \item Structured content types
  \item Content type behaviors
  \item State based workflows
  \item Roles and permissions
  \item All types of portlets
  \end{itemize}
  \end{column}
  \begin{column}{0.45\textwidth}
  \begin{itemize}
  \setlength{\itemsep}{1em}
  \mdseries
  \item Content rules on events
  \bfseries
  \item Restricted templates
  \item Restricted Python scripts
  \mdseries
  \item Static frontend resources
  \item Diazo transform rules
  \item \ldots
  \end{itemize}
  \end{column}
  \end{columns}
\end{frame}

\begin{frame}[plain,c]
  \frametitle{Restricted Python?}
  \Large
  \bfseries
  ”Restricted Python provides a safety net for programmers who don't know the details of the Zope/Plone security model.
  \par
  \vspace{1em}
  It is important that you understand that the safety net is not perfect: it is not adequate to protect your site from coding by an untrusted user.”
  \par
  \vspace{1em}
  \normalsize
  \raggedleft
  \mdseries
  \textit{-- Steve McMahon, collectice.ambidexterity README}
\end{frame}

\begin{frame}[plain,t]
  \frametitle{Issues with Restricted Python}
  \stepcounter{beamerpauses}
  \par
  Not only ponies and unicorns...
  \par
  \vspace{0.33cm}
  \begin{itemize}[<+->]
  \setlength{\itemsep}{1em}
  \item API is \texttt{restrictedTraversed}, not imported
  \item API is not always complete
  \item API is not always up-to-date
  \item API is scattered around the ecosystem
  \item \href{https://api.plone.org/}{\texttt{plone.api}} is not currently designed or available by default to be called from Restricted Python
  \end{itemize}
\end{frame}

\begin{frame}[plain,c]
\centering\includegraphics[trim=0 0 0 0,clip,width=0.5\paperwidth,]{images/practical-plone.jpg}
\end{frame}

\begin{frame}[plain,c]
  \huge
  \bfseries
  \centering
  \vspace{1cm}
  Products.DocFinderTab
\end{frame}

\begin{frame}[plain,t]
  \vspace{8em}
  \huge
  \bfseries
  \centering
  ”through-the-web”
  \par
  $=$
  \par
  technical debt
\end{frame}

\section{Advanced themes}

\begin{frame}[plain,t]
  \frametitle{Advanced Plone themes}
  \stepcounter{beamerpauses}
  \vspace{1em}
  \begin{itemize}[<+->]
  \setlength{\itemsep}{1em}
  \item Theme with any supported customizations
  \item Editable using Plone theme editor
  \item Rollbacks with Zope2 undo form
  \item Zip-exportable and importable from theme editor
  \item Supports ”TTW” and ”file system" development
  \item Exports can be version controlled
  \item Exports can be acceptance tested
  \end{itemize}
\end{frame}

\begin{frame}[plain,t]
  \frametitle{collective.themesitesetup}
  \stepcounter{beamerpauses}
  \begin{columns}[onlytextwidth,t]
  \begin{column}{0.45\textwidth}
  \begin{itemize}[<+->]
  \setlength{\itemsep}{1em}
  \item[] \hspace{-16pt} Theme activation or update
  \item Imports GS profile steps
  \item Updates DX models
  \item Registers permissions
  \item Registers l10m messages
  \item Copies resources into \texttt{portal\_resources}
  \end{itemize}
  \end{column}
  \begin{column}{0.45\textwidth}
  \begin{itemize}[<+->]
  \setlength{\itemsep}{1em}
  \item[] \hspace{-16pt} Theme deactivation
  \item Imports GS profile steps
  \item Unregisters added custom permissions
  \item Unregisters added custom l10n messages
  \end{itemize}
  \begin{itemize}[<+->]
  \setlength{\itemsep}{1em}
  \item[] \hspace{-16pt} After ”TTW” development
  \item \texttt{@@export-site-setup}
  \end{itemize}
  \end{column}
  \end{columns}
\end{frame}

\begin{frame}[plain,t]
  \frametitle{collective.themefragments}
  \stepcounter{beamerpauses}
  \begin{itemize}[<+->]
  \setlength{\itemsep}{1em}
  \item[] \hspace{-16pt} View templates: \texttt{./fragments/foobar.pt}
  \item Can be injected as fragments using Diazo rules
  \item Can be configured as a default view for any content type
  \item Can be used as a local view by setting \texttt{layout} attribute
  \item[] \hspace{-16pt} Python scripts: \texttt{./fragments/foobar.py}
  \item Can provide view methods for view templates with matching base name
  (\texttt{tal:define="data view/getData"})
  \end{itemize}
\end{frame}

\begin{frame}[plain,t]
  \vspace{8em}
  \huge
  \bfseries
  \centering
  Faster iterations
  \par
  $=$
  \par
  More iterations
\end{frame}

\section{Example}

\begin{frame}[plain,c]
  \huge
  \bfseries
  \centering
  \vspace{1cm}
  Let's get our hands dirty\dots
\end{frame}

\begin{frame}[plain,t]
  \frametitle{Creating Wall of images}
  \begin{itemize}[<+->]
  \setlength{\itemsep}{1em}
  \item Initial configuration was made through Plone site setup
  \item Configuration was exported into a new theme using the export view
  \texttt{++theme++\ldots/@@export-site-setup}
  \item Theme was zip-exported from theming control panel
  \item Content type schemas were exported from Dexterity editor
  \item Development was completed on a regular file system buildout
  using file system resources directory
  \end{itemize}
\end{frame}

\begin{frame}[plain,fragile,t]
  \frametitle{Structure of Wall of images}
  \begin{minted}[breaklines]{bash}
./bundles/
./fragments/
./install/types/
./install/workflows/
./install/
./locales/LC_MESSAGES/*/
./models/
./index.html
./manifest.cfg
./preview.png
./rules.xml
./scripts.js
./styles.css
  \end{minted}
\end{frame}

\begin{frame}[plain,fragile,t]
\frametitle{Custom frontend bundles\hfill$1/2$}
\begin{minted}[breaklines]{bash}
./bundles/imagesloaded.pkgd.min.css
./bundles/imagesloaded.pkgd.min.js
./bundles/masonry.pkgd.min.css
./bundles/masonry.pkgd.min.js
\end{minted}
\begin{minted}[breaklines]{js}
(function() { var require, define;
// ...
// AMD packaged JS distributions must be wrapped
// so that Plone require.js define is undefined
// during their load
// ...
})();
\end{minted}
\end{frame}

\begin{frame}[plain,fragile,t]
\frametitle{Custom frontend bundles\hfill$2/2$}
\begin{minted}[breaklines]{bash}
./install/registry.xml
\end{minted}
\begin{minted}[breaklines]{xml}
<records prefix="plone.bundles/imagesloaded-js"
         interface="...interfaces.IBundleRegistry">
  <value key="depends">plone</value>
  <value key="jscompilation">++theme++...js</value>
  <value key="csscompilation">++theme++...css</value>
  <value key="last_compilation">2017-10-06 00:00:00
  </value>
  <value key="compile">False</value>
  <value key="enabled">True</value>
</records>
\end{minted}
\end{frame}

\begin{frame}[plain,fragile,t]
\frametitle{Custom content types\hfill$1/2$}
\begin{minted}[breaklines]{bash}
./install/types/wall_of_images.xml
./install/types/wall_of_images_image.xml
./install/types.xml
\end{minted}
\begin{minted}[breaklines]{xml}
<object name="portal_types">
 <object name="wall_of_images"
         meta_type="Dexterity FTI"/>
 <object name="wall_of_images_image"
         meta_type="Dexterity FTI"/>
</object>
\end{minted}
\begin{minted}[breaklines]{bash}
./models/wall_of_images.xml
./models/wall_of_images_image.xml
\end{minted}
\end{frame}

\begin{frame}[plain,fragile,t]
\frametitle{Custom content types\hfill$2/2$}
\begin{minted}[breaklines]{xml}
<model xmlns:...="..." i18n:domain="plone">
  <schema>
    <field
        name="title" type="zope.schema.TextLine">
      <title i18n:translate="">Title</title>
    </field>
    <field
        name="image"
        type="plone.namedfile.field.NamedBlobImage">
      <title i18n:translate="">Image</title>
    </field>
  </schema>
</model>
\end{minted}
\end{frame}

\begin{frame}[plain,fragile,t]
\frametitle{Custom views\hfill$1/3$}
\begin{minted}[breaklines]{xml}
<html xmlns="http://www.w3.org/1999/xhtml"
      xmlns:...="..."
      lang="en"
      metal:use-macro=".../macros/master"
      i18n:domain="plone">
<body>
<metal:main fill-slot="main">
  <metal:content-core define-macro="content-core">
  </metal:content-core>
</metal:main>
</body>
</html>
\end{minted}
\end{frame}

\begin{frame}[plain,fragile,t]
\frametitle{Custom views\hfill$2/3$}
\begin{minted}[breaklines]{bash}
./fragments/wall-of-images.pt
\end{minted}
\begin{minted}[breaklines]{xml}
<div class="wall-of-images container-fluid"
     tal:define="items context/@@contentlisting">
  <tal:image tal:repeat="item items">
    <img tal:define="obj item/getObject;
        scale_func obj/@@images;
        scaled_image python:scale_func.scale('image', scale='preview')"
      tal:replace="structure python:scaled_image.tag()"
      tal:on-error="string:error" />
  </tal:image>
</div>
\end{minted}
\end{frame}

\begin{frame}[plain,fragile,t]
\frametitle{Custom views\hfill$3/3$}
\begin{minted}[breaklines]{bash}
./install/types/wall_of_images.pt
\end{minted}
\begin{minted}[breaklines]{xml}
...
<property name="default_view">
    ++themefragment++wall-of-images</property>
<property name="view_methods">
  <element value="++themefragment++wall-of-images"/>
</property>
...
\end{minted}
\vspace{0.33cm}
Any themefragment can be configured as content object view by manually setting
\texttt{layout} property of the content object.
\end{frame}

\begin{frame}[plain,fragile,t]
\frametitle{Custom l10n messages}
\begin{minted}[breaklines]{bash}
./locales/fi/LC_MESSAGES/plone.po
./locales/en/LC_MESSAGES/plone.po
\end{minted}
\vspace{0cm}
\begin{minted}[breaklines]{po}
msgid "Close from submissions"
msgstr "Sulje osallistuminen"

msgid "Open for submissions"
msgstr "Avaa osallistumiselle"
\end{minted}
\vspace{0.25cm}
When theme is developed on file system, normal i18n tools can be used
for messages extraction and catalog synchronization (e.g. i18ndude).
\end{frame}

\begin{frame}[plain,fragile,t]
\frametitle{Custom permissions}
\begin{minted}[breaklines]{bash}
./manifest.cfg
\end{minted}
\begin{minted}[breaklines]{ini}
[theme:genericsetup]
permissions =
    demotheme.addImage  Wall of Images: Add Image
\end{minted}
\begin{minted}[breaklines]{bash}
./install/types/wall_of_images_image.xml
\end{minted}
\begin{minted}[breaklines]{xml}
<object name="wall_of_images_image" ...="..">
  ...
  <property name="add_permission">
      demotheme.addImage</property>
  ...
</object>
\end{minted}
\end{frame}

\begin{frame}[plain,fragile,t]
\frametitle{Custom workflows}
\begin{minted}[breaklines]{bash}
./install/workflows.xml
./install/simple_publication_with_submission_workflow/
./install/wall_of_images_workflow/
\end{minted}
\begin{minted}[breaklines]{xml}
<type type_id="wall_of_images">
  <bound-workflow workflow_id="..."/>
</type>
\end{minted}
\begin{minted}[breaklines]{xml}
<dc-workflow workflow_id="...">
  ...
  <permission>Wall of Images: Add Image</permission>
  ...
</dc-workflow>
\end{minted}
\end{frame}

\begin{frame}[plain,fragile,t]
\frametitle{Custom content rules}
\begin{minted}[breaklines]{bash}
./install/contentrules.xml
\end{minted}
\begin{minted}[breaklines]{xml}
<contentrules>
  <rule name="rule-image-thank-you"
        title="Thank visitor from submission"
        cascading="False"
        description="Thanks visitor after submission"
        event="...IObjectAddedEvent"
        stop-after="False"
        enabled="True">...</rule>
  <assignment
      name="rule-image-thank-you" bubbles="True"
      enabled="True" location=""/>
</contentrules>
\end{minted}
\end{frame}

\begin{frame}[plain,fragile,t]
\frametitle{Final touch with Diazo-bundles}
\begin{minted}[breaklines]{bash}
./manifest.cfg
\end{minted}
\begin{minted}[breaklines]{ini}
production-css = /++theme++demotheme/styles.css
production-js = /++theme++demotheme/scripts.js
\end{minted}
\begin{minted}[breaklines]{bash}
./scripts.js
\end{minted}
\begin{minted}[breaklines]{js}
jQuery(function($) {
  $('.wall-of-images').imagesLoaded(function() {
    $('.wall-of-images').masonry({
      itemSelector: 'img',
      percentPosition: true
    });
  });
});
\end{minted}
\end{frame}

\section{The End}

{
\usebackgroundtemplate{\includegraphics[trim=0 0 0 0,clip,height=\paperheight]{images/questions.jpg}}%
\begin{frame}[plain,t]
  \setbeamercolor{background canvas}{bg=black}
  \vfill
  \centering
  \Huge
  \bfseries
  \lsstyle
  \contour{white}{Questions?}
  \par
  \vspace{0.5cm}
  \Large
  \contourlength{1.00pt}
  \contour{white}{github.com/datakurre/ploneconf2017}
\end{frame}
}

\begin{frame}[plain,t]
  \frametitle{What about Mosaic?}
  \begin{itemize}
  \setlength{\itemsep}{1em}
  \item Theme site setup can populate Mosaic site and content layout
  resource directories from theme
  \item Theme fragment tile for Plone Mosaic
  can render selected fragment with
  \vspace{1em}
  \begin{itemize}
  \setlength{\itemsep}{1em}
  \item fragment-specific readable title
  \item fragment-specific XML schema based configuration form
  \item fragment-specific view permission
  \item fragment-specific caching rule
  \end{itemize}
  \end{itemize}
\end{frame}


\begin{frame}[plain,t]
  \frametitle{What about Webpack?}
  \begin{itemize}
  \setlength{\itemsep}{1em}
  \item Bundles in theme can be built with Webpack
  \item Diazo bundle in theme can be built with Webpack
  \item All frontend resources can be built with Webpack
  \vspace{1em}
  \begin{itemize}
  \setlength{\itemsep}{1em}
  \item \href{https://github.com/collective/plonetheme.webpacktemplate}{\texttt{plonetheme.webpacktemplate}}
  \item \href{https://www.npmjs.com/package/plonetheme-webpack-plugin}{\texttt{plonetheme-webpack-plugin}}
  \item \href{https://www.npmjs.com/package/plonetheme-upload}{\texttt{plonetheme-upload}}
  \end{itemize}
  \end{itemize}
\end{frame}

\begin{frame}[plain,fragile,t]
\frametitle{Bonus: Custom JS widgets}
\begin{minted}[breaklines]{bash}
./models/my_content_type.xml
\end{minted}
\begin{minted}[breaklines]{xml}
...
<field name="focuspoint"
       type="zope.schema.BytesLine">
  <form:widget
      type="z3c.form.browser.text.TextFieldWidget">
    <klass>text-widget pat-focuspoint-widget</klass>
  </form:widget>
  <title i18n:translate="">Image focus point</title>
  <required>false</required>
</field>
...
\end{minted}
\end{frame}

\end{document}
